
%%%%%%%%%%%%%%%%%%%%%%%%%%%%%%%%%%%%%%%%%%%%%%%%%%%%%%%%%%%%%%%%%%%%%%%%%%%%%%%%
%% Preambles
%%%%%%%%%%%%%%%%%%%%%%%%%%%%%%%%%%%%%%%%%%%%%%%%%%%%%%%%%%%%%%%%%%%%%%%%%%%%%%%%
<<<<<<< Updated upstream
\documentclass[a4paper, 12pt, DIV11, BCOR5mm, tikz]{scrartcl}
=======
\documentclass[a4paper, 11pt, DIV11, BCOR5mm]{scrartcl}
>>>>>>> Stashed changes
\usepackage[ngerman]{babel}
\usepackage[utf8]{inputenc}
\usepackage{siunitx}
\usepackage{fancyhdr}
\usepackage{float}
\usepackage{graphicx}
\usepackage[dvipsnames]{xcolor}
\usepackage{hyperref}
\usepackage{ragged2e}
\usepackage{titlesec}
\usepackage{listings}
\usepackage{framed}
\usepackage[linguistics]{forest}

%%%%%%%%%%%%%%%%%%%%%%%%%%%%%%%%%%%%%%%%%%%%%%%%%%%%%%%%%%%%%%%%%%%%%%%%%%%%%%%%
%% General configurations
%%%%%%%%%%%%%%%%%%%%%%%%%%%%%%%%%%%%%%%%%%%%%%%%%%%%%%%%%%%%%%%%%%%%%%%%%%%%%%%%
\sisetup{locale=DE}
\graphicspath{ {Bilder/} }
\setlength{\parskip}{12 pt}
\titleformat*{\section}{\LARGE\bfseries}
\titleformat*{\subsection}{\large\bfseries}

%%%%%%%%%%%%%%%%%%%%%%%%%%%%%%%%%%%%%%%%%%%%%%%%%%%%%%%%%%%%%%%%%%%%%%%%%%%%%%%%
%% Counter definitions
%%%%%%%%%%%%%%%%%%%%%%%%%%%%%%%%%%%%%%%%%%%%%%%%%%%%%%%%%%%%%%%%%%%%%%%%%%%%%%%%
\newcounter{imagecounter}

%%%%%%%%%%%%%%%%%%%%%%%%%%%%%%%%%%%%%%%%%%%%%%%%%%%%%%%%%%%%%%%%%%%%%%%%%%%%%%%%
%% New commands definitions
%%%%%%%%%%%%%%%%%%%%%%%%%%%%%%%%%%%%%%%%%%%%%%%%%%%%%%%%%%%%%%%%%%%%%%%%%%%%%%%%
\newcommand{\bfh}{\textsc{Berner Fachhochschule}}
\newcommand{\modul}[1]{\textsc{#1}}
\newcommand{\thema}{Bitcoin}
\newcommand{\code}[1]{{\ttfamily #1}}
\newcommand{\fdate}[1]{\emph{#1\newline\newline}}

\newcommand{\query}[2]{
    \code{?- #1}\\
    \indent
    \code{#2}
}

\NewDocumentCommand\semester{O{Herbstsemester}m}{#1 #2}

%%%%%%%%%%%%%%%%%%%%%%%%%%%%%%%%%%%%%%%%%%%%%%%%%%%%%%%%%%%%%%%%%%%%%%%%%%%%%%%%
%% Color definitions
%%%%%%%%%%%%%%%%%%%%%%%%%%%%%%%%%%%%%%%%%%%%%%%%%%%%%%%%%%%%%%%%%%%%%%%%%%%%%%%%
\definecolor{mygray}{rgb}{0.5,0.5,0.5}
\definecolor{mymauve}{rgb}{0.58,0,0.82}
\definecolor{myred}{HTML}{C70039}
\definecolor{mygreen}{HTML}{C5D200}%{A9C52F}
\definecolor{mydarkgreen}{HTML}{51710A}
\definecolor{myblue}{HTML}{07617D}
\definecolor{mywhite}{HTML}{F5F5F5}
\definecolor{mydarkgray}{HTML}{424242}

%%%%%%%%%%%%%%%%%%%%%%%%%%%%%%%%%%%%%%%%%%%%%%%%%%%%%%%%%%%%%%%%%%%%%%%%%%%%%%%%
%% Renewed commands
%%%%%%%%%%%%%%%%%%%%%%%%%%%%%%%%%%%%%%%%%%%%%%%%%%%%%%%%%%%%%%%%%%%%%%%%%%%%%%%%
\renewcommand{\headrulewidth}{0.4pt}
\renewcommand{\footrulewidth}{0.4pt}

%%%%%%%%%%%%%%%%%%%%%%%%%%%%%%%%%%%%%%%%%%%%%%%%%%%%%%%%%%%%%%%%%%%%%%%%%%%%%%%%
%% Hyperlinks configuration
%%%%%%%%%%%%%%%%%%%%%%%%%%%%%%%%%%%%%%%%%%%%%%%%%%%%%%%%%%%%%%%%%%%%%%%%%%%%%%%%
\hypersetup{
    bookmarks=true,         % show bookmarks bar?
    unicode=false,          % non-Latin characters in Acrobat’s bookmarks
    pdftoolbar=true,        % show Acrobat’s toolbar?
    pdfmenubar=true,        % show Acrobat’s menu?
    pdffitwindow=false,     % window fit to page when opened
    pdfstartview={FitH},    % fits the width of the page to the window
    pdftitle={My title},    % title
    pdfauthor={Author},     % author
    pdfsubject={Subject},   % subject of the document
    pdfcreator={Creator},   % creator of the document
    pdfproducer={Producer}, % producer of the document
    pdfkeywords={keyword1, key2, key3}, % list of keywords
    pdfnewwindow=true,      % links in new PDF window
    colorlinks=true,        % false: boxed links; true: colored links
    linkcolor=myblue,        % color of internal links (change box color with linkbordercolor)
    citecolor=mygreen,      % color of links to bibliography
    filecolor=myred,        % color of file links
    urlcolor=myblue         % color of external links
}

%%%%%%%%%%%%%%%%%%%%%%%%%%%%%%%%%%%%%%%%%%%%%%%%%%%%%%%%%%%%%%%%%%%%%%%%%%%%%%%%
%% Listings configuration
%%%%%%%%%%%%%%%%%%%%%%%%%%%%%%%%%%%%%%%%%%%%%%%%%%%%%%%%%%%%%%%%%%%%%%%%%%%%%%%%
\lstset{%
    backgroundcolor=\color{mywhite},   % choose the background color; you must add \usepackage{color} or \usepackage{xcolor}; should come as last argument
    basicstyle=\ttfamily,            % the size of the fonts that are used for the code
    breakatwhitespace=false,         % sets if automatic breaks should only happen at whitespace
    breaklines=true,                 % sets automatic line breaking
    captionpos=b,                    % sets the caption-position to bottom
    commentstyle=\color{mygreen},    % comment style
    deletekeywords={...},            % if you want to delete keywords from the given language
    escapeinside={\%*}{*)},          % if you want to add LaTeX within your code
    extendedchars=true,              % lets you use non-ASCII characters; for 8-bits encodings only, does not work with UTF-8
    frame=l,	                     % adds a frame around the code
    framerule=1pt,                   % controls the width of the rule
    keepspaces=true,                 % keeps spaces in text, useful for keeping indentation of code (possibly needs columns=flexible)
    keywordstyle=\color{myblue},       % keyword style
    language=Prolog,                 % the language of the code
    morekeywords={*, use_module, append},                % if you want to add more keywords to the set
    numbers=left,                    % where to put the line-numbers; possible values are (none, left, right)
    numbersep=8pt,                   % how far the line-numbers are from the code
    numberstyle=\footnotesize\color{mydarkgray}, % the style that is used for the line-numbers
    rulecolor=\color{mydarkgreen},         % if not set, the frame-color may be changed on line-breaks within not-black text (e.g. comments (green here))
    showspaces=false,                % show spaces everywhere adding particular underscores; it overrides 'showstringspaces'
    showstringspaces=false,          % underline spaces within strings only
    showtabs=false,                  % show tabs within strings adding particular underscores
    stepnumber=1,                    % the step between two line-numbers. If it's 1, each line will be numbered
    stringstyle=\color{mymauve},     % string literal style
    tabsize=2,	                     % sets default tabsize to 2 spaces
    title=\lstname,                  % show the filename of files included with \lstinputlisting; also try caption instead of title
    xleftmargin=\parindent           % sets the position relative to left margin
}
%%%%%%%%%%%%%%%%%%%%%%%%%%%%%%%%%%%%%%%%%%%%%%%%%%%%%%%%%%%%%%%%%%%%%%%%%%%%%%%%
%% Start of document
%%%%%%%%%%%%%%%%%%%%%%%%%%%%%%%%%%%%%%%%%%%%%%%%%%%%%%%%%%%%%%%%%%%%%%%%%%%%%%%%
\begin{document}

%%%%%%%%%%%%%%%%%%%%%%%%%%%%%%%%%%%%%%%%%%%%%%%%%%%%%%%%%%%%%%%%%%%%%%%%%%%%%%%%
%% Cover page
%%%%%%%%%%%%%%%%%%%%%%%%%%%%%%%%%%%%%%%%%%%%%%%%%%%%%%%%%%%%%%%%%%%%%%%%%%%%%%%%
\pagenumbering{gobble}
\thispagestyle{empty}

\begin{titlepage}
    \begin{figure}
        \begin{minipage}{0.3\textwidth}
            \begin{figure}[H]
                \includegraphics[scale=0.5]{BFH_Logo}
            \end{figure}
        \end{minipage}
        \hfill
        \begin{minipage}[t]{0.5\textwidth}
            \begin{flushright}
                Berner Fachhochschule\\
                Technik und Informatik\\
                \bigskip
                Herbstsemester 16/17\\
                \bigskip
                Modul: BTI7311 - Informatik Seminar\\
                Betreuer: Kai Brünnler\\
            \end{flushright}
        \end{minipage}
    \end{figure}
    \setlength{\textfloatsep}{4cm}
    {\centering
    {\scshape\large Seminararbeit \par}
    {\scshape\LARGE\thema\par}}
    \vfill
    \begin{figure}[!hb]
        \begin{minipage}[t][][b]{0.4\textwidth}
            von\\
            Carlos Arauz\\
            5. Semester Informatik\\
            Bärenfelserstrasse 9\\
            4057 Basel\\
            arauc1@bfh.ch\\
            Tel.: +41 78 920 41 89
        \end{minipage}
        \begin{minipage}[t][][b]{0.4\textwidth}
            {\color{white}von}\\
            Angelo Campanile\\
            5. Semester Informatik\\
            Bottigenstrasse 2A\\
            3018 Bern\\
            campa1@bfh.ch\\
            Tel.: +41 79 696 09 16
        \end{minipage}
    \end{figure}
\end{titlepage}
\newpage

\pagenumbering{roman}       % set the page number to roman numbers
\setcounter{page}{2}        % start the page numbering from 2

%%%%%%%%%%%%%%%%%%%%%%%%%%%%%%%%%%%%%%%%%%%%%%%%%%%%%%%%%%%%%%%%%%%%%%%%%%%%%%%%
%% Preface
%%%%%%%%%%%%%%%%%%%%%%%%%%%%%%%%%%%%%%%%%%%%%%%%%%%%%%%%%%%%%%%%%%%%%%%%%%%%%%%%
\noindent
{\scshape\LARGE Vorworte \par}
\vspace{10mm}
\noindent
\justify
Diese Seminararbeit führt die logische Programmierung mithilfe der Programmiersprache Prolog ein.
Angenommen wird, dass der Leser bereits über ein allgemeines Wissen der Programmierung verfügt,
weil diese Arbeit dem Programmierparadigma der logischen Programmierung Nachdruck verleiht.

\noindent
Die Beispiele, welche in diesem Dokument vorgestellt werden, sind mit SWI-Prolog \mbox{(\url{http://www.swi-prolog.org/})}
getestet worden und können mit den meisten anderen Prolog-Systemen gleich gut funktionieren. Eine kurze Einleitung der
Installierung von Prolog wird auch beschreibt, nachher werden nur die Quellcodes bzw. die Rückgabewerte der Abfragen angezeigt.

%\cite{bibo1}
%\begin{thebibliography}{9}
%\bibitem{bibo1}
%Michel Goossens, Frank Mittelbach, and Alexander Samarin.
%\textit{The \LaTeX\ Companion}.
%Addison-Wesley, Reading, Massachusetts, 1993.
%\end{thebibliography}
\newpage

%%%%%%%%%%%%%%%%%%%%%%%%%%%%%%%%%%%%%%%%%%%%%%%%%%%%%%%%%%%%%%%%%%%%%%%%%%%%%%%%
%% Index
%%%%%%%%%%%%%%%%%%%%%%%%%%%%%%%%%%%%%%%%%%%%%%%%%%%%%%%%%%%%%%%%%%%%%%%%%%%%%%%%
\tableofcontents
\newpage

% \listoffigures
% \newpage

%%%%%%%%%%%%%%%%%%%%%%%%%%%%%%%%%%%%%%%%%%%%%%%%%%%%%%%%%%%%%%%%%%%%%%%%%%%%%%%%
%% Header and Footer
%%%%%%%%%%%%%%%%%%%%%%%%%%%%%%%%%%%%%%%%%%%%%%%%%%%%%%%%%%%%%%%%%%%%%%%%%%%%%%%%
\pagestyle{fancy}
\fancyhead[L]{\bfh}
\fancyhead[R]{\modul{Informatik Seminar}}
\fancyfoot[L]{\thema}
\fancyfoot[R]{\semester{2016/2017}}

\pagenumbering{arabic}  % sets the page numbers to arabic and restarts it

%%%%%%%%%%%%%%%%%%%%%%%%%%%%%%%%%%%%%%%%%%%%%%%%%%%%%%%%%%%%%%%%%%%%%%%%%%%%%%%%
%% Capitel 000
%%%%%%%%%%%%%%%%%%%%%%%%%%%%%%%%%%%%%%%%%%%%%%%%%%%%%%%%%%%%%%%%%%%%%%%%%%%%%%%%

\vspace*{1mm}
\section*{Einleitung}
\addcontentsline{toc}{section}{Einleitung}
\vspace{-10mm}
\noindent\rule{0.8\textwidth}{0.4pt}

\vspace{5mm}

\noindent
In der letzen Dekade hat sich der Begriff Kryptowährung schnell verbreitet. Unter Kryptowährung
(auch als Kryptogeld bezeichnet) versteht man digitales Geld, welches durch die Anwendung
der Prinzipien der Kryptographie ein verteiltes, dezentrales und sicheres digitales Zahlungssystem
verwirklicht.

\noindent
Seit der Erscheinung des ersten öffentlich gehandelten Kryptogelds, des Bitcoins, wurde eine grosse
Menge weiterer Kryptowährungen implementiert und über 3000 von denen sind zurzeit in Verwendung. In
den folgenden Kapiteln werden wir uns mit dem seit 2009 gehandelten Bitcoin beschäftigen.

\noindent
Aber damit die ersten geh versuche mit Bitcoin gelingen, muss man sich zuerst die Frage beantworten
können \emph{\dq Was ist Bitcoin eigentlich?\dq}


\newpage


%%%%%%%%%%%%%%%%%%%%%%%%%%%%%%%%%%%%%%%%%%%%%%%%%%%%%%%%%%%%%%%%%%%%%%%%%%%%%%%%
%% Capitel 001
%%%%%%%%%%%%%%%%%%%%%%%%%%%%%%%%%%%%%%%%%%%%%%%%%%%%%%%%%%%%%%%%%%%%%%%%%%%%%%%%


%%%%%%%%%%%%%%%%%%%%%%%%%%%%%%%%%%%%%%%%%%%%%%%%%%%%%%%%%%%%%%%%%%%%%%%%%%%%%%%%
%% Capitel 002
%%%%%%%%%%%%%%%%%%%%%%%%%%%%%%%%%%%%%%%%%%%%%%%%%%%%%%%%%%%%%%%%%%%%%%%%%%%%%%%%


%%%%%%%%%%%%%%%%%%%%%%%%%%%%%%%%%%%%%%%%%%%%%%%%%%%%%%%%%%%%%%%%%%%%%%%%%%%%%%%%
%% Capitel 003
%%%%%%%%%%%%%%%%%%%%%%%%%%%%%%%%%%%%%%%%%%%%%%%%%%%%%%%%%%%%%%%%%%%%%%%%%%%%%%%%

\vspace*{1mm}
\section*{Schlussworte}
\addcontentsline{toc}{section}{Schlussworte}
\vspace{-10mm}
\noindent\rule{0.8\textwidth}{0.4pt}

\noindent
Die Seminararbeit zum Thema Logische Programmierung mit Prolog hat uns einen tiefen Einblick in der Welt der Logischen Programmierung verschafft. Gerade zu Beginn war das Verständnis der Logischen Programmierung nicht so einfach zu verstehen. Zudem war Prolog Programme zu Programmieren ziemlich seltsam. Doch nach vielen Tutorials und Übungen wurden wir immer besser, schneller und effizienter in der Programmierung. Es war extrem spannend herauszufinden, was für Möglichkeiten die Logische Programmierung bietet, also wo sind ihre Stärken und Schwächen.

\noindent
Prolog kann durch seine Turing-Vollständigkeit jedes Programmierproblem lösen, jedoch ist es für das einte oder andere Programm weniger sinnvoll Prolog zu verwenden. Durch unsere Recherchen haben wir festgestellt, dass Prolog vor allem für Programme geeignet ist, in welchem man sehr viel über das Problem weiss. Dadurch können die Daten dann in Fakten, Regeln, usw. eingeschrieben werden, um das Problem dann zu lösen. Es ist natürlich auch möglich eine Graphische Oberfläche mit Prolog zu programmieren, jedoch ist dies sehr kompliziert und Zeitaufwändig, dass es dafür bessere Programmiersprachen gibt. Auf der einen Seite hat Prolog eine einfache Semantik und ist leicht durch die Fakten und Regeln modularisierbar, jedoch ist es auch ziemlich Fehleranfällig, da auf Datentypen verzichtet wird,  und man leicht unbeabsichtigt Endlosschleifen erstellen kann.

\noindent
Die Arbeit hat viel Zeit in Anspruch genommen, doch am Ende hat es sich gelohnt.
Während dieser Arbeit haben wir viel Wissen mitnehmen können, was wir in Zukunft sicherlich einmal gebrauchen können.
Dies ist nun das Ende unserer ersten Informatik-Seminar Arbeit, uns ist bewusst, dass dies nur einen Überblick der Logischen Programmierung darstellt und dass eine vertiefte Recherche sehr spannend wäre.
Vielen Dank an die Dozenten welche uns die Möglichkeit gegeben haben dieses Thema erarbeiten zu dürfen.


\nocite{*}
\newpage
\addcontentsline{toc}{section}{Literatur}
\bibliography{Bibliografie/literatur}
\bibliographystyle{alpha}
\end{document}
