\section*{Vor- und Nachteile}
\addcontentsline{toc}{section}{Vor- und Nachteile}
\vspace{-10mm}
\noindent\rule{0.8\textwidth}{0.4pt}
\subsection*{Vorteile}
\addcontentsline{toc}{subsection}{Vorteile}
\vspace{-10mm}
\noindent\rule{0.8\textwidth}{0.4pt}
\begin{description}
  \item[Zahlungsfreiheit] Bitcoin erlaubt den Benutzern, beliebige Geldbeträge
  ohne Verzögerung zu senden oder zu empfagen und zwar überall auf der Welt und
  zu jeder Zeit. Im Gegensatz zu Banken stellt Bitcoin keine Limits, wie viel Geld
  der Benutzer am Tag oder im Monat senden kann oder wie viel Geld er in seinem
  Konto haben kann. Der Benutzer kann die 365 Tage Zahlungen realisieren bzw.
  erhalten und es gibt keine Landesgrenzen. Alles, was jemand braucht, ist, Bitcoins
  und ein entprechender Bitcoin-Client.
  \item[Niedrige Gebühren] Bitcoin-Zahlungen werden derzeit mit keiner oder einer
  sehr geringen Gebühr verarbeitet. Benutzer können Gebühren zu Tranksaktionen
  hinzufügen, was dazu führt, dass die entsprechende Transaktion schneller bestätigt wird.
  \item[Weniger Risiken für Händler] Da Bitcoin-Tranksaktionen sicher, unumkehrbar sind und
  keine Daten des Käufers beinhalten, werden Händler vor Verlüsten geschützt, weil
  Buchungen nicht rückgängig gemacht werden können (zum Beispiel). Anderes Beispiel
  ist, dass Händler Kunden, welche keine Kreditkarten zur Verfügung haben, erreichen
  können; ohne Kreditkarten zu arbeiten, verringt das Risiko auf Betrug.
  \item[Sicherheit und Kontrolle] Da das Bitcoin-Protokoll auf mathematischen Algorithmen
  basiert und wird von keiner bestimmten Organisation gesteuert, sind Bitcoin-Transak-tionen sicher,
  diese brauchen keine persönlichen Informationen der Bitcoin-Nutzer, um durchgeführt zu werden
  und somit ist die Identität des Bitcoin-Ntzers vor Identitätsdiebstahl geschützt. Es gibt keine Bitcoin-Banke und daher
  hat der Benutzer voller Kontrolle auf sein Geld. Den Bitcoin-Nutzern stehen zwei Möglichkeiten
  zur verfügung, ihr Geld zu schützen: \emph{Backups} und \emph{Verschlüsselung}.
  \item[Transparenz und Neutralität] Das Bitcoin-Protokoll ist sicher verschlüsselt, daher kann
  keine Einzelperson oder Organisation es kontrollieren oder manipulieren. Aus diesem Grund können die Bitcoind-Nutzer
  dem Kern von Bitcoin vertrauen, dass er völlig neutral, transparent und vorhersehbar ist.

\end{description}
\subsection*{Nachteile}
\addcontentsline{toc}{subsection}{Nachteile}
\vspace{-10mm}
\noindent\rule{0.8\textwidth}{0.4pt}
\begin{description}
  \item[Item] some text
\end{description}
