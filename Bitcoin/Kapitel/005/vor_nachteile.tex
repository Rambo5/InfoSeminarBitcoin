\section*{Vor- und Nachteile}
\addcontentsline{toc}{section}{Vor- und Nachteile}
\vspace{-10mm}
\noindent\rule{0.8\textwidth}{0.4pt}
\subsection*{Vorteile}
\addcontentsline{toc}{subsection}{Vorteile}

\begin{description}
  \item[Zahlungsfreiheit] Bitcoin erlaubt den Benutzern, beliebige Geldbeträge
  ohne Verzögerung zu senden oder zu empfangen und zwar überall auf der Welt und
  zu jeder Zeit. Im Gegensatz zu Banken stellt Bitcoin keine Limits, wie viel Geld
  der Benutzer am Tag oder im Monat senden kann oder wie viel Geld er in seinem
  Konto haben kann. Der Benutzer kann die 365 Tage Zahlungen realisieren bzw.
  erhalten und es gibt keine Landesgrenzen. Alles, was jemand braucht, sind Bitcoins
  und ein entsprechender Bitcoin-Client.
  \item[Niedrige Gebühren] Bitcoin-Zahlungen werden derzeit mit keiner oder einer
  sehr geringen Gebühr verarbeitet. Benutzer können Gebühren zu Transkaktionen
  hinzufügen, was dazu führt, dass die entsprechende Transaktion schneller bestätigt wird.
  \item[Weniger Risiken für Händler] Da Bitcoin-Tranksaktionen sicher, unumkehrbar sind und
  keine Daten des Käufers beinhalten, werden Händler vor Verlusten geschützt, weil
  Buchungen nicht rückgängig gemacht werden können (zum Beispiel). Anderes Beispiel
  ist, dass Händler Kunden, welche keine Kreditkarten zur Verfügung haben, erreichen
  können; ohne Kreditkarten zu arbeiten, verringert das Risiko auf Betrug.
  \item[Sicherheit und Kontrolle] Da das Bitcoin-Protokoll auf mathematischen Algorithmen
  basiert und wird von keiner bestimmten Organisation gesteuert, sind Bitcoin-Transaktionen sicher,
  diese brauchen keine persönlichen Informationen der Bitcoin-Nutzer, um durchgeführt zu werden
  und somit ist die Identität des Bitcoin-Nutzers vor Identitätsdiebstahl geschützt. Es gibt keine Bitcoin-Banken und daher
  hat der Benutzer volle Kontrolle auf sein Geld. Den Bitcoin-Nutzern stehen zwei Möglichkeiten
  zur Verfügung, ihr Geld zu schützen: \emph{Backups} und \emph{Verschlüsselung}.
  \item[Transparenz und Neutralität] Das Bitcoin-Protokoll ist sicher verschlüsselt, daher kann
  keine Einzelperson oder Organisation es kontrollieren oder manipulieren. Aus diesem Grund können die Bitcoin-Nutzer
  dem Kern von Bitcoin vertrauen, dass er völlig neutral, transparent und vorhersehbar ist.

\end{description}
\subsection*{Nachteile}
\addcontentsline{toc}{subsection}{Nachteile}

\begin{description}
  \item[Grad der Akzeptanz] Vielen Menschen sind Bitcoin noch unbekannt oder sie haben davon
  gehört, möchten aber es nicht ausprobieren, weil sie nicht verstehen, wie sie funktionieren
  oder was Bitcoins eigentlich sind. Obwohl jeden Tag immer mehr Firmen Bitcoins als Zahlungsmethode
  akzeptieren, verbleibt die Liste noch klein und um vom Netzwerkeffekt profitieren zu können, müsste
  diese Liste wachsen.
  \item[Volatilität] Da Bitcoins ihre volle Potenzial wegen einer Mangel an Akzeptanz noch nicht
  erreicht haben, hängt ihr Wert davon ab, wie gefordert Bitcoins sind. Mit anderen Worten, wenn Menschen
  viel Interesse an Bitcoins haben und bereit sind, einen gewissen Betrag für sie zu zahlen, steigt der
  Wert des Bitcoin und wenn nicht, sinkt er.
  \item[Laufende Entwicklung] Die Bitcoin Software und Bitcoins selbst sind noch relativ jung. Das heisst,
  dass sich viele Funktionen noch in der Entwicklungsphase befinden. Neue Funktionen, Werkzeuge und Dienste
  werden auch eingeführt, um Bitcoin sicherer zu machen und um zu machen, dass Bitcoin eine grössere Masse
  erreicht.

\end{description}
\newpage
