\vspace*{1mm}
\section*{Einleitung}
\addcontentsline{toc}{section}{Einleitung}
\vspace{-10mm}
\noindent\rule{0.8\textwidth}{0.4pt}

\vspace{5mm}

\noindent
Damit die ersten geh versuche mit Bitcoin gelingen, muss man sich zuerst die Frage beantworten können \emph{\dq Was ist Bitcoin eigentlich? \dq}

\noindent
Bitcoin ist nichts anderes als ein Computerprogramm, das von Satoshi Nakamoto entwickelt wurde und an dem 3. Januar 2009 ins Leben gerufen wurde.
Es ist eine digitale Währung die noch sehr jung ist, noch in den Kinderschuhen steckt und laufend weiterentwickelt wird.
Es beruht auf einem dezentralen Peer-to-Peer Zahlungssystem, was bedeutet, dass keine zentrale stelle benötigt wird um die Geldbeträge zu überweisen.
Natürlich gibt es eine Kontrolle wie Transaktionen kontrolliert werden können, doch dafür wird die in den Code einprogrammierte Kryptografie verwendet.
Bitcoin wird nicht durch eine Person kontrolliert, sondern durch alle Nutzer auf der Welt.

\noindent
Bitcoin ist nur als Programm vorhanden und nicht in irgend etwas materiellem. Gleichwohl konnte sich der Wert von einem Bitcoin Jahr für Jahr vergrössern.
Wieso aber halten wir den Bitcoin für etwas Wertvolles obwohl es ihn in der realen Welt gar nicht gibt? Nun die Menschen geben Sachen eine Wert, wenn
man glaubt und das Vertrauen hat, dass etwas einen bestimmten Wert hat und andere das auch so sehen. Da man also angefangen hat mit Bitcoins zahlen zu können,
hat es somit eine Wert erhalten, da man sich somit verschiedene Sachen kaufen kann. Und solange es Menschen gibt, die den Bitcoin für Wertvoll halten, wird sich
an dem dem System nicht viel ändern.

\noindent
Da Bitcoin ein Open-Source Projekt ist, kann also jeder den Code des Systems einsehen. Dies hat den Vorteil, dass Fehler oder Sicherheitslücken schnell
gefunden und behoben werden. Somit kann auch jeder die Software einsehen, sie aber nicht den anderen erzwingen, da jeder selbst eintscheiden kann welche
Version sie benutzen wollen. Es ist jedoch nötig, dass alle Nutzer die gleiche Software brauchen, die die gleichen Regeln enthält. Nur so kann man mit
Bitcoins handeln. Und da das System nur dann funktioniert, wenn man sich einig ist, ist es natürlich von grossem interesse, dass das so bleibt.

\noindent
Um Bitcoin zu benutzen, braucht man sich nur das Bitcoin-Client Programm herunterzuladen und schon kann man sich Bitcoins kaufen. Das System wird von
vielen geschätzt und ist für viele Personen interessant, da man, um mit Bitcoins zu handeln nur einen Computer und eine Internet verbindung benötigt.
Somit sind Perosnen, welche in einem Gebiet leben, wo man keine Bank in der nähe hat, nicht von dem System ausgeschlossen.

\noindent
Das System setzt einen grossen Wert auf anonymität, was bedeutet, dass sich jeder ein \emph{\dq Bitcoin-Konto \dq} erstellen kann, da es ja nur ein
Programm ist, das man sich herunterladen muss. Deshalb werden Bitcoins natürlich auch für Illegale Zwecke verwendet, wie z.B. im Darknet. Zudem ist eine
Geldüberweisung nicht mehr umkehrbar, wie z.B. bei einer Kredirkarten buchung.

\noindent
Damit nicht eine Inflation entstehen kann wie bei einer realen Währung, es also zu viel Geld im umlauf hat. Das System beugt dies vor indem es die gesammte Menge
an exstierenden Bitcoins schon vordefiniert hat auf 21 Millionen. Die Gesamtzahl der derzeit existierenden Bitcoins beträgt ca. 16 Millionen. Und jeden Tag werden
neue generiert, bis diese 21 Millionen erreicht wurden.



\newpage
