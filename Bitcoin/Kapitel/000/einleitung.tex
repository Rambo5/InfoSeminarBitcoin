\vspace*{1mm}
\section*{Einleitung}
\addcontentsline{toc}{section}{Einleitung}
\vspace{-10mm}
\noindent\rule{0.8\textwidth}{0.4pt}

\vspace{5mm}
\noindent
Die logische Programmierung ist ein Programmierparadigma, welches Teil des Paradigmas der deklarativen
Programmierung ist. Andere Subparadigmen, die auch der deklarativen Programmierung angehören, sind: funktionale
Programmierung, Constraintprogrammierung, domänenspezifische Sprachen (\emph{engl. domain-specific language},
kurz DSL) Programme und hybride Programmiersprachen.\par

\noindent
Wichtige logische Programmiersprachenfamilien umfassen Prolog, Antwortmengenprogrammierung (\emph{engl. Answer set
programming}, kurz ASP) und Datalog. In diesen drei Sprachen werden Regeln in Form von Klauseln geschrieben:
\begin{lstlisting}
% H wenn B1 und ... und Bn
H :- B1, ..., Bn.
\end{lstlisting}
\vspace{-8mm}
\noindent
Das Verwenden von mathematischer Logik, um Computerprogramme darzustellen und auszuführen, ist auch eine Eigenschaft
des \href{https://de.wikipedia.org/wiki/Lambda-Kalkül}{Lambda-Kalküls}, welcher in den 30er Jahren von
\href{https://de.wikipedia.org/wiki/Alonzo_Church}{Alonzo Church} entwickelt wurde. Der erste Vorschlag der Verwendung
von der Klauselnform der Logik, um Computeranwendungen darzustellen, wurde jedoch von Cordell Green \cite{cgreen} gemacht.\par

\noindent
Logische Programmierung in ihrer jetzigen Form geht auf Diskussionen über deklarative gegen prozedurale Darstellungen des
Wissens künstlicher Intelligenz in den späten 1960er und frühen 1970er Jahren zurück. Befürworter der deklarativen
Repräsentationen arbeiteten fleissig an der Standford University zusammen mit John McCarthy, Bertram Raphael und
Cordell Green, und in Edinburgh, mit John Alan Robinson, einem akademischen Besucher aus der Syracuse University, Pat Hayes
und \href{https://en.wikipedia.org/wiki/Robert_Kowalski}{Robert Kowalski}, wer mit \href{https://de.wikipedia.org/wiki/Alain_Colmerauer}{Alain Colmerauer}
in Marseille zusammenarbeitete. Colmerauer entwickelte dann die Ideen des Entwurfs und der Implementierung von Prolog.\par

\noindent
Prolog liess die Programmiersprachen ALF, Fril, Gödel, Mercury, Oz, Ciao, Visual Prolog, XSB, und $\lambda$Prolog, sowie
eine Menge von parallelen logischen Programmiersprachen, Constraint-Logik-Programmiersprachen und Datalog entstehen.\par

\noindent
Die \href{http://www.cs.nmsu.edu/ALP/the-association-for-logic-programming/alp-history/}{Association for Logic Programming (ALP)} wurde 1986 gegründet.
Ihre Aufgabe ist es, die Entwicklung der logischen Programmierung zu fördern.
