\section*{Ökonomie und Gesetz}
\addcontentsline{toc}{section}{Ökonomie und Gesetz}
\vspace{-10mm}
\noindent\rule{0.8\textwidth}{0.4pt}

\vspace{5mm}

\noindent
Wie bereits erwähnt wurde, werden neue Bitcoins durch ein in Konkurrenz stehendes und
dezentrales Verfahren namens \emph{Mining} erzeugt. Sie werden mit einer abnehmenden
und vorhersagbaren Rate generiert, dies führt dazu, dass die Anzahl neuer Bitcoins,
welche jedes Jahr produziert werden, automatisch über die Zeit halbiert, bis die Gesamtsumme
aller Bitcoins 21 Millionen erreicht.

\noindent
Der Wert des Bitcoins variiert über die Zeit. 2010 war der Wert eines Bitcoins weniger
als \$0.01 und im November 2013 erreichte der Bitcoin den bisher maximalen Wert
von \$1'250 : 1 BTC. Dass Bitcoins lediglich von einem mathematischen Algorithmus
generiert werden, lässt die Frage entstehen \emph{\dq wieso haben Bitcoins
überhaupt einen Wert?\dq}.

\noindent
Eine Währung (zum Beispiel der Dollar) beruht auf physikalischen Eigenschaften oder dem Vertauen
auf zentrale Autoritäten. Der Bitcoin hingegen basiert auf Eigenschaften der Mathematik und
er verfügt über die Charakteristik vom Geld (unter anderem Übertragbarkeit, Portabilität und
Erkennbarkeit). Dies gibt Bitcoins ihren Wert, weil sie auch in Form vom Geld genutzt werden
können und daher und wie bei allen anderen Währungen bildet sich der Wert des Bitcoins direkt und
nur durch die Menschen, die ihn als Bezahlung akzeptieren.

\noindent
Was aber der Preis eines Bitcoins bestimmt, ist, Angebot und Nachfrage. Wenn die Nachfrage nach
Bitcoins steigt, steigt der Preis, wenn sie sinkt, sinkt ebenfalls der Preis. Dass Bitcoin zur Zeit
noch ein relativ kleiner Markt ist, führt dazu, dass keine grösseren Geldbeträge nötig sind, um den
Marktpreis zu steigern oder zu verringern. Dies macht den Preis eines Bitcoins sehr volatil.

\noindent
Das Bitcoin-Protokoll kann nicht modifiziert werden ohne das Zusammenwirken fast all seiner Benutzer,
denn sie wählen, welche Software sie verwenden. Dies macht den Versuch, einer lokalen Autorität
spezielle Rechte zuzuteilen, praktisch unanwendbar. Wenn keine Autorität Bitcoins reguliert werden kann,
kann die Frage gestellt werden, ob sie wirklich legal sind. Bisher sind Bitcoins in keinem Gerichtsstand
illegal, in manchen Ländern (wie Argentinien und Russland) werden Fremdwährungen beschränkt oder
komplett verboten.

\noindent
Ein weiterer Aspekt der Bitcoins liegt hauptsächlich darin, dass Bitcoin Geld ist und Geld wurde
schon immer für legale und illegale Zwecke benutzt. Da Bitcoins für private und unumkehrbare Zahlungen
verwendet werden kann, wurden Vorwürfe gemacht, weil das Bitcoins attraktiver für Krimineller machen
könnte.
