\section*{Ökonomie und Gesetz}
\addcontentsline{toc}{section}{Ökonomie und Gesetz}
\vspace{-10mm}
\noindent\rule{0.8\textwidth}{0.4pt}

\vspace{5mm}

\noindent
Wie bereits erwähnt wurde, werden neue Bitcoins durch ein in Konkurrenz stehendes und
dezentrales Verfahren namens \emph{Mining} erzeugt. Sie werden mit einer abnehmenden
und vorhersagbaren Rate generiert, dies führt dazu, dass die Anzahl neuer Bitcoins,
welche jedes Jahr produziert werden, automatisch über die Zeit halbiert, bis die Gesammtsumme
aller Bitcoins 21 Millionen erreicht.

Im November 2013 erreichte der Bitcoin den bisher maximalen Wert von \$1'250 : 1 BTC
Dass Bitcoins lediglich von einem matematischen Algorithmus generiert werden, lässt
die Frage entstehen \emph{\dq wieso haben }

\newpage
