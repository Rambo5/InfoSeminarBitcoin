\vspace*{1mm}
\subsection*{Hash-Funktionen}
\addcontentsline{toc}{subsection}{Hash-Funktionen}
\vspace{-10mm}
\noindent\rule{0.8\textwidth}{0.4pt}

\vspace{5mm}

\noindent
Eine Hash Funktion ist nichts anderes als eine Funktion, die eine Message als Input in einen Output wandelt. Dabei muss man beachten, dass der Output immer eine definierte länge hat.

\noindent
Eine Kryptographische Hash-funktion muss einige Sicherheitspunkte beachten:

\noindent
1. Effizienz:
Eine Kryptographische Hash-Funktion sollte möglichst effizient sein. Sie sollte also einen Hashwert einer Message schnell und effizient generieren können.

\noindent
2. Kollisions sicher
Der Output einer Kryptographischen Hash-Funktion sollte so einmalig wie nur möglich sein. Also wenn zwei verschiedene Inputs eingegeben werden, müssen immer verschiedene Outputs generiert werden.
Dies ist vorallem beim Bitcoin System wichtig, da ansonsten das System Probleme bekommen könnte.

\noindent
3. Zufälligkeit
Ein Output sollte möglichst Zufällig aussehen. Man sollte nicht durch den Output erkennen welchen Input eingegeben wurde, was ein wesentlicher teil des Bitcoin Systems ist.

\noindent
4. Zurückhaltung von Informationen
Man sollte versuchen wo wenig Informationen wie nur möglich Preiszugeben. Das sind Informationen wie z.B. ob die Message eine gerade oder ungerade Anzahl Elemente enthält.
Das könnte nämlich zu einem Sicherheitsproblem führen in Systemen.



\newpage
