\vspace*{1mm}
\subsection*{Block-Chain}
\addcontentsline{toc}{subsection}{Block-Chain}
\vspace{-10mm}
\noindent\rule{0.8\textwidth}{0.4pt}

\vspace{5mm}

\noindent
Ein weiterer wichtiger Term ohne das, die Bitcoin währung nicht funktionieren würde, ist die Blockchain.
Die Blockchain, also auf Deutsch \emph{\dq Block - Kette\dq}, ist eine auflistung aller Transaktionen, die jemals im Bitcoinsystem ausgeführt wurde.
Die momentane grösse der Blockchain beträgt momentan um die 100 GB und wird von Tag zu Tag grösser.

\noindent
Wenn wir in der Realen Welt eine Geldüberweisung machen wollen, müssen wir das über einen Drittanbieter wie z.B. einer Bank erledigen. Mit der Blockchain
versucht man dies zu umgehen. Zudem erreicht man mit der Blockchain, dass die Transaktionen viel schneller überwiesen werden als mittels einer Bank z.B. Zudem
spart man sich viel Geld, da die Gebühren im Bitcoinsystem viel niedriger sind als bei einer Bank.

\noindent
Also Die Blockchain ist eine Liste von zusammenhängenden Transaktionen, welche nicht an  einer Zentralen stelle gespeichert sind, da das System ja dezentral halten möchte,
von der jeder Benutzer im Netzwerk eine Kopie davon hat.


\newpage
