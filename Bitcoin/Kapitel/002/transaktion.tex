\vspace*{1mm}
\subsection*{Transaktion}
\addcontentsline{toc}{subsection}{Transaktion}
\vspace{-10mm}
\noindent\rule{0.8\textwidth}{0.4pt}

\vspace{5mm}

\noindent
In diesem Abschnitt versuchen wir zu erklären, wie genau eine solche Transaktion, also eine Geldüberweisung, Schritt für Schritt durchgeführt wird.
Dies zeigen wir anhand eines Beispiels bei dem ein Benutzer einem anderen Benutzer eine Menge von Bitcoins sendet. Diese Benutzer nennen wir zur Verständigung
\emph{\dq Sender \dq} und \emph{\dq Empfänger \dq}.

\noindent
Damit man überhaupt Bitcoins transferieren können, brauchen wir den Bitcoin client zu installieren. Dieser Client kann man sich auf dieser Seite herunterladen: \emph{\dq https://bitcoin.org/de/ \dq}.
Durch herunterlade der Software hat man sich jetzt eine Wallet erstellt, also ein Portemonnaie für die Bitcoins. Leider kann zu diesem Zeitpunkt noch nicht mit Bitcoins handeln.
Dafür muss man sich nämlich eine Adresse generieren lassen und diese ist einmalig. Mit dieser Adresse kann man nun Bitcoins transferieren.
Man kann sich beliebig viele Adressen generieren lassen, die zur gleichen wallet gehören. Es wird sogar empfohlen für jede Transaktion eine neue Adresse zu gebrauchen zum Schutz der Anonymität.
Beim Erstellen dieser Adresse wird automatisch ein Private- und ein Public-key generiert. Der Private-key ist der einzige Zugang zu der Adresse, sollte man den also verlieren, hat man keine Möglichkeit mehr
auf die Bitcoins zuzugreifen. Der Private key ist auch stark mit der Adresse gekoppelt, also mit dem Private-key kann man die Adresse berechnen aber nicht umgekehrt. Das bedeutet sollte jemand fremdes in den
Besitz dieses Private-keys gelangen, so findet er auch die Adresse und hat somit Zugang zu all sich darin befindenden Bitcoins. Es gibt keine Möglichkeit Bitcoins ohne Private-key zurück zu erhalten.
Diese Bitcoins bleiben zwar im System erhalten, jedoch können sie nicht mehr abgerufen werden. Der Public-key wird dazu verwendet, um zu validieren, ob die Transaktion wirklich von diesem Benutzer
erstellt wurde. Dies wird in einem späteren Kapitel genauer erläutert.

\noindent
Da in unserer Wallet noch keine Bitcoins vorhanden sind, können wir auch keine transferieren. Es gibt verschiedene Möglichkeiten an Bitcoins zu gelangen.
Man kann z.B. Bitcoins als Zahlung von jemandem akzeptieren und somit Bitcoins erhalten oder man geht auf einer Bitcoin-Webseite, wo man mit realem Geld Bitcoins kaufen kann.
Es gibt auch die Möglichkeit, dass man Bitcoins durch sogenanntes \emph{\dq Minen \dq} erhält, das wird in einem späteren Kapitel noch behandelt.

\noindent
Auf dem GUI des Bitcoin Programms kann der Sender nun die Adresse des Empfängers angeben und den Betrag der Bitcoins, der übertragen werden soll.
Sobald er auf Senden drückt, wird die Transaktion dem System gesendet. Uns interessiert aber was hinter dem GUI passiert und wie sich diese Transaktion zusammenstellt.
Eine Transaktion besteht aus verschiedenen Elementen. Nämlich aus einer Message, einer Signature und einem Public-key. Diese werden nun einzeln genauer erläutert:

\noindent
Die Message:
Die Message besteht selber aus verschiedenen Elementen die alle Informationen enthalten. Damit es möglich ist dass der Sender dem Empfänger eine Gewisse Anzahl Bitcoins senden kann, muss natürlich
der Sender diese Bitcoins auch haben. Nehmen wir an, der Sender, der 100 Bitcoins besitzt, möchte dem Empfänger 90 Bitcoins senden, dann wird zuerst geschaut von wo der Sender diese Bitcoins eigentlich hat. Dies ist möglich da
alle Transaktionen die jemals gemacht wurden in der Blockchain gespeichert wurden. Also sehen wir dort z.B. drei Transaktionen mit je 30 Bitcoins. diese drei Transaktionen werden nun je als Hash in der
Message gespeichert und werden als Input dieser Message angeschaut. Als Output werden auch wieder Transaktionen als Hashwert eingefügt. Dazu zählt die Information dass der Sender dem Empfänger 90 Bitcoins senden
will. Diese Information braucht die Anzahl der zu sendenden Bitcoins und den Public-key des Empfängers. Als zweiten Output Wert dieser Message wird folgendes gebraucht, die Information wie viele Bitcoins wieder
zurück zum Sender gehen. Da der Sender ja 100 Bitcoins besitzt aber nur 90 senden will, wird noch angegeben, wie viele Bitcoins wieder zurück zum Sender gelangen. Dazu braucht man den Public-key des Senders,
und die Menge der zurückgehenden Bitcoins. In diesem Fall werden wir annehmen, dass das 9 Bitcoins sein werden. Die Frage, die man sich jetzt sofort stellt ist, was passiert mit dem letzten verbleibenden
Bitcoin? Nun dieser wird als Transaktionsgebühr verwendet und geht an den Benutzer, der die Transaktion löst, aber dazu später mehr. Nun haben wir alle Elemente der Message. Und der Hash dieser Message, ist
zugleich der Name der Transaktion, wie auch der Wert der weiterverwendet wird.

\noindent
Die Signature:
Die Signature ist genau gleich wie eine Unterschrift in der realen Welt. Mit der Unterschrift bestätigt man also, dass irgendein Dokument von sich selber und niemand anderes erstellt wurde und gültig ist.
Bei der Signature beim Bitcoin ist das genau das gleiche. Man signiert eine Transaktion, um deren Wahrheitswert zu belegen und zu definieren, dass diese Transaktion zu einem selber gehört. Die Signature
setzt sich aus dem Hash der Message und dem Private-key zusammen. Diese zwei Elemente generieren einen weiteren Hash, den wir die Signature nennen. Diese Signature ist nun genau dieser Transaktion
und dieser Message gebunden. Also sie ist nicht nur dem Sender zugeteilt, was ja eigentlich logisch sein muss, da der Private-key des Senders gebraucht wird, sondern auch der Message selber. Denn wird
dem Private-key eine andere Message hinzugefügt, so ändert sich auch die Signature.

\noindent
Der Public-key
Der Public-key wird nur verwendet um zu verifizieren ob die Message und die Signature wirklich zueinander gehören und somit die Transaktion verifiziert ist.

\noindent
Da die Transaktion nun erstellt wurde kommen wir nun zum Teil, wo die Miner ins Spiel kommen.

\vspace*{1mm}
\subsubsection*{Mining}
\addcontentsline{toc}{subsubsection}{Mining}
\vspace{-10mm}
\noindent\rule{0.8\textwidth}{0.4pt}

\vspace{5mm}

\noindent
Mining wird in diesem System gebraucht, um sicherzustellen, dass alle Nutzer synchronisiert sind. Dazu werden alle in den letzten 10 Minuten erstellten Transaktionen in einen Transaktionsblock gelegt.
Nun werden die gesammelten Daten in einen Kryptographischen Hash gewandelt. Dieser Hash beinhaltet Informationen über den letzten gelösten Block, sowie eine Zufallsgenerierte Zahl, die
\emph{\dq Nounce \dq} genannt wird. Nun haben die Miner die Aufgabe, die Nounce herauszufinden, die diese Hashvalue verursacht. Das System ist so programmiert dass die Nounce eine Value
generieren muss, die mit einer bestimmten Anzahl nullen beginnen muss. Das Ratsel ist auf die Schwierigkeit eingestellt, dass man etwa 10minuten für einen Transaktionblock braucht um die Lösung
zu finden. Es ist klar, dass immer neue technologien entwickelt werden und schnellere Computer existieren werden, deshalb werden somit die Rätsel immer schneller gelöst werden. Um dieses Problem vorzubeugen,
ändert man einfach die Schwierigkeit der Nounce, damit der vorgang immer etwa 10 Minuten dauert. Der Miner, der das Rätsel gelöst hat, erhält als Belohnung ein paar Bitcoins.

\noindent
Früher brauchten die Miner Ihren Prozessor um die Rätsel zu lösen, danach hatte man gemerkt, dass mit Graphikkarten die Rätsel viel schneller gelöst werden konnte und das ging so weiter,dass
Heutzutage sogar spezialisierte Hardware existiert, die extra dafür entwickelt wurden, Bitcoins zu minen. Dadurch wurde es immer schwieriger als Individuum durch Minen Bitcoins zu erhalten.
Deshalb wurden sogenannte Mining-Pools erschaffen, wo viele Miner zusammen versuchen das Rätsel zu lösen und somit die Wahrscheinlichkeit höher ist dass man das Rätsel löst. Die Belohnung wird dann
gleichermassen an alle Miner im Pool aufgeteilt.

\noindent
Das Mining trägt dazu bei, die Sicherheit im System aufrecht zu erhalten, indem es durch die Lotterie es verhindert, das irgend jemand einen Transaction Block in die Blockchain hinzufügen kann.
Da Jeder Block in der Blockchain informationen aus dem vorherigen Block enthält, sichert sich das System vor änderungen ab. Sollte also jemand versuchen einen Block in der Blockchain
zu ändern, dann würde das sofort auffallen und als ungültig erklärt werden, da der Hashwert nicht stimmen würde. Zudem verhindert es, dass Transaktionsblöcke rückgängig gemacht werden, da man
sonst alle Transaktionen in diesem Block neu schreiben müsste.

\noindent
Also ein Miner verifiziert, ob eine Transaktionüberhaupt gültig ist indem er den Public-key mit der Message und der Signature vergleicht. Trifft dies zu, so wird die Transaktion in einen Block gespeichert,
wo alle Transaktionen gespeichert werden, welche in den Letzten 10 Minuten aufgegeben wurden.

\noindent
Die Transaktion wird in ein Kryptographisches Hash Rätsel umgewandelt. Dabei werden Informationen aus dem letzten block in der Blockchain gebraucht, den Informationen der Transaction, und einer Nounce.
Eine Nounce ist eine Zufallsgenerierte Zahl, welche zum Hash Rätsel hinzugefügt wird, dass dieser so Zufällig wie möglich ist. Dieses rätsel generiert nun ein Resultat dem sogenannten \emph{\dq Target \dq}.
Das Target ist so aufgebaut, dass die Lösung z.B. mit einer gewissen Anzahl Nullen beginnen muss. Die Miner wählen nun die Transaktion aus, welche sie lösen möchten und versuchen einen Wert, nennen wir ihn
\emph{\dq Key \dq}, herauszufinden, welches es schafft, dass die Lösung kleiner als der Target ist.

\noindent
Findet nun jemand die Lösung, also den Key, der das Rätsel löst, so werden alle Transaktionen dieses Blocks zu der Blockchain hinzugefügt. So erhält dieser Miner die Gewinnsumme der Bitcoins und Broadcastet
seine Lösung allen anderen Nutzer im Netzwerk. Diese Verifizieren, ob die Lösung korrekt ist und fügen diese in Ihre Blockchain hinzu.

\noindent
Wie wir nun verhindert, dass jemand mit einem extrem starken Rechner, jede Transaktion lösen kann, und somit eigentlich sleber bestimmen kann, welche Transaktion als nächste genommen werden muss.
Nun dies wird so gemacht, dass jeder Versuch diesen Key zu finden als ein \emph{\dq Vote \dq} angesehen wird, und so kann man verhindern, dass jemand 1000 mal mehr votet als andere.



\newpage




\newpage
