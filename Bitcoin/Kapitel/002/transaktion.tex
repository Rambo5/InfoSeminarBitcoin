\vspace*{1mm}
\subsection*{Transaktion}
\addcontentsline{toc}{subsection}{Transaktion}
\vspace{-10mm}
\noindent\rule{0.8\textwidth}{0.4pt}

\vspace{5mm}

\noindent
In diesem Abschnitt versuchen wir zu erklären, wie genau eine solche Transaktion, also eine Geldüberweisung, Schritt für Schritt durchgeführt wird.
Dies zeigen wir anhand eines Beispiels bei dem ein Benutzer einem anderen Benutzer eine Menge von Bitcoins sendet.

\noindent
Damit man überhaupt Bitcoins transferieren können, brauchen wir den Bitcoin client zu installieren. Dieser Client kann man sich auf dieser Seite herunterladen: \emph{\dq https://bitcoin.org/de/ \dq}.
Durch herunterlade der Software hat man sich jetzt eine Wallet erstellt, also ein Porte-monnaie für die Bitcoins. Leider kann zu diesem Zeitpunkt noch nicht mit Bitcoins handeln.
Dafür muss man sich nämlich eine Adresse generieren lassen und diese ist einmalig. Mit dieser Adresse kann man nun Bitcoins transferieren.
Man kann sich beliebig viele Adressen generieren lassen, die zur gleichen wallet gehören. Es wird sogar empfohlen für jede Transaktion eine neue Adresse zu gebrauchen zum Schutz der Anonymität.
Beim erstellen dieser Adresse wird automatisch ein Private- und ein Public-key generiert. Der Private-key ist der einzige Zugang zu der Adresse, sollte man den also verlieren, hat man keine möglichkeit mehr
auf die Bitcoins zuzugreifen. Der Private key ist auch stark mit der Adresse gekoppelt, also mit dem Private-key kann man die Adresse berechnen aber nicht umgekehrt. Das bedeutet sollte jemand fremdes in den
Besitz dieses Private-keys gelangen, so findet er auch die Adresse und hat somit zugang zu all sich darin befindenden Bitcoins. Es gibt keine möglichkeit Bitcoins ohne Private-key zurück zu erhalten.
Diese Bitcoins bleiben zwar im System erhalten, jedoch können sie nicht mehr abgerufen werden.

\noindent





\newpage
