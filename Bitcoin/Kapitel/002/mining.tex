\vspace*{1mm}
\subsubsection*{Mining}
\addcontentsline{toc}{subsubsection}{Mining}
\vspace{-10mm}
\noindent\rule{0.8\textwidth}{0.4pt}

\vspace{5mm}

\noindent
Mining wird in diesem System gebraucht, um sicherzustellen, dass alle Nutzer synchronisiert sind. Dazu werden alle in den letzten 10 Minuten erstellten Transaktionen in einen Transaktionsblock gelegt.
Nun werden die gesammelten Daten in einen Kryptographischen Hash gewandelt. Dieser Hash beinhaltet Informationen über den letzten gelösten Block, sowie eine Zufallsgenerierte Zahl, die
\emph{\dq Nounce \dq} genannt wird. Nun haben die Miner die Aufgabe, die Nounce herauszufinden, die diese Hashvalue verursacht. Das System ist so programmiert dass die Nounce eine Value
generieren muss, die mit einer bestimmten Anzahl nullen beginnen muss. Das Ratsel ist auf die Schwierigkeit eingestellt, dass man etwa 10minuten für einen Transaktionblock braucht um die Lösung
zu finden. Es ist klar, dass immer neue technologien entwickelt werden und schnellere Computer existieren werden, deshalb werden somit die Rätsel immer schneller gelöst werden. Um dieses Problem vorzubeugen,
ändert man einfach die Schwierigkeit der Nounce, damit der vorgang immer etwa 10 Minuten dauert. Der Miner, der das Rätsel gelöst hat, erhält als Belohnung ein paar Bitcoins.

\noindent
Früher brauchten die Miner Ihren Prozessor um die Rätsel zu lösen, danach hatte man gemerkt, dass mit Graphikkarten die Rätsel viel schneller gelöst werden konnte und das ging so weiter,dass
Heutzutage sogar spezialisierte Hardware existiert, die extra dafür entwickelt wurden, Bitcoins zu minen. Dadurch wurde es immer schwieriger als Individuum durch Minen Bitcoins zu erhalten.
Deshalb wurden sogenannte Mining-Pools erschaffen, wo viele Miner zusammen versuchen das Rätsel zu lösen und somit die Wahrscheinlichkeit höher ist dass man das Rätsel löst. Die Belohnung wird dann
gleichermassen an alle Miner im Pool aufgeteilt.




\newpage
